\documentclass[12pt]{report}
\usepackage{suthesis}

% -- Imports --
% (general libraries)
\usepackage{times,latexsym,amsfonts,amssymb,amsmath,graphicx,url,bbm,rotating}
\usepackage{multirow,hhline,stmaryrd,bussproofs,mathtools,siunitx}
\usepackage{booktabs,xcolor,csquotes,calligra}
\usepackage{mhchem}
% (custom libraries)
\usepackage{afterpage}
\usepackage{longtable}
\usepackage{fitch}
% (inline references)
\usepackage{natbib}
\usepackage{tabularx}
\usepackage[hidelinks]{hyperref}
\hypersetup{
    colorlinks=true,
    citecolor=magenta,
    linkcolor=.,
    urlcolor=blue
}

\usepackage{epigraph}
\renewcommand{\epigraphsize}{\normalsize}
\setlength{\epigraphwidth}{0.9\textwidth}
\usepackage{subcaption}

% (tikz)
\usepackage{soul}
\definecolor{light-yellow}{RGB}{255, 255, 153}
\sethlcolor{light-yellow}
\usepackage{tikz}
\usepackage{tikz-dependency,pifont}
\usetikzlibrary{shapes.arrows,chains,positioning,automata,trees,calc}
\usetikzlibrary{patterns,matrix}
\usetikzlibrary{decorations.pathmorphing,decorations.markings}
% (print algorithms)
\usepackage[ruled,lined,linesnumbered]{algorithm2e}
% (custom)
\input{std-macros.tex}
\input{macros.tex}

% (paper compilation hacks)
\def\newcite#1{\citet{#1}}
\def\cite#1{\citep{#1}}
%\def\newcite#1{\textcite{#1}}
%\def\cite#1{\autocite{#1}}
\definecolor{darkblue}{rgb}{0.0,0.0,0.4}


% Common hyphenations
\hyphenation{Text-Runner}
\hyphenation{Verb-Ocean}
\hyphenation{Raj-pur-kar}

%\bibliographystyle{plainnat}


% Comments
\usepackage{xspace}
\usepackage{xargs} % commandx
\usepackage[colorinlistoftodos,prependcaption,textsize=tiny]{todonotes}
\usepackage{marginnote}
\usepackage{color}
\definecolor{darkgreen}{RGB}{0,100,0}

% Inline comments useful for tables and figures.
\newcommandx{\icmtl}[2][1=]{\todo[inline]{DC: #2}\xspace}
\newcommandx{\icmtm}[2][1=]{\todo[inline]{CM: #2}\xspace}

% Comments for other places.
\newcommandx{\cmtl}[2][1=]{\todo[linecolor=blue,backgroundcolor=blue!10,bordercolor=blue,#1]{DC: #2}\xspace}
\newcommandx{\cmtm}[2][1=]{\todo[linecolor=red,backgroundcolor=red!10,bordercolor=red,#1]{CM: #2}\xspace}

\newcommand\cmb[1]{\marginpar{\tiny\raggedright\textcolor{blue}{\textsf{ DC\@: #1}}}}
\newcommand\cmm[1]{\marginpar{\tiny\raggedright\textcolor{red}{\textsf{\bfseries CM\@: #1}}}}

\usepackage{enumerate}

\setcounter{secnumdepth}{3}

\usepackage{footnote}
\makesavenoteenv{tabular}
\makesavenoteenv{table}

\usepackage{xpinyin}

% -- Document --
\begin{document}

% UCL Logo
\vspace*{-5cm}
\hspace*{-1.35cm}
\makebox[\textwidth]{\includegraphics[width=\paperwidth]{img/_UCL_Banner-a4-port-black.eps}}
\vfill % Elastic empty space filler

% Title
\title{An Image Recognition Algorithm Approach to Distinguishing and Categorising the Small-Boat Fleet in the Gulf of California}
\author{UCL Candidate Code: KNCM2}
\principaladviser{Santiago Suarez De LaFuente}

% Preface
\beforepreface
%!TEX root = thesis.tex

\prefacesection{Abstract}

Teaching machines to recognize and even understand satellite images is one of the most elusive and persistent challenges in artificial intelligence. This thesis tackles satellite image recognition: using the data and models needed to train models to identify small boats and measure their length. On the one hand, detecting small boats in satellite images is a major test of artificial intelligence algorithms. But, on the other hand, if we can build efficient object detection models, it will become a key technology for building carbon inventories in the shipping industry.

This thesis focuses on convolutional neural network models: a class of neural network models built on top of convolutional operations. These end-to-end neural models are more effective in learning media content (e.g., texts, images, and videos) and substantially improve performance on all object detection benchmarks than traditional feature-based models.

This thesis consists of two parts. In the first part, our goal is to cover every aspect of object detection algorithm and show our efforts in building effective object detection models and, more importantly, to understand what convolutional neural network models have actually learned. In the second part of this thesis, I show my detection results and analyze the various possibilities behind the results.

In particular, I discuss two new research directions at the end of this thesis: 1) whether a CAD model can replace the data from the training model, and 2) whether the algorithm has the ability (memory and understanding) to identify the same ship that appears at different times. I believe that they hold great promise for future satellite image analysis.
%!TEX root = thesis.tex
\prefacesection{Acknowledgments}


A special thanks to my supervisor Dr Santiago Suarez de la Fuente. He was always very insightful and visionary about the field, but he was also very detail-oriented and understood the nature of the issues very well. More importantly, Dr Suarez de la Fuente is an extremely caring and supportive mentor. He always gave positive feedback in each weekly meeting, and I couldn't ask for more.

Finally, I want to thank other academic colleagues inside and outside UCL, my fellow schoolmates, family and friends. Without them, this project would not be finished smoothly: Andrea Grech La Rosa, UCL Research Computing, Edward Gryspeerdt, Tom Lutherborrow, Jeff Jinfeng Guo, Mia Stieglitz-Courtney, Shiyi, Hao, Ju, and Shunying.

\afterpreface
\hypersetup{linkcolor=magenta}


% -- Sections --
% Introduction
\chapter{Introduction}
\label{chapter:Introduction}
%!TEX root = thesis.tex

\section{Motivation}
\subsection{Energy Crisis, Resources and Climate Change}
The energy crisis is one of the most essential and critical crises in the 21st century. Due to the growth of population and the increase of energy intensity per capita, a shortage of energy supply has occurred frequently in recent years, which often causes an energy crisis, usually involving scarcity of oil, electricity or other natural resources. 

Nowadays, non-renewable resource still consists of a large proportion in the energy system. However, according to the BP Statistical Review of World Energy \& Ember (see Figure~\ref{intro_renewable}), the share of electricity production from renewables has continued growing since 2007. In 2020, the share of electricity production from renewables was around 29\% (see Figure~\ref{intro_renewable}). In other words, non-renewables are still the majority sources of electricity production today, although their share is shrinking.

\begin{figure}[!t]
\center
\includegraphics[scale=0.12]{img/intro_renewable.png}
\longcaption{Share of electricity production from renewables.}{The share of electricity production from renewables increased from around 18\% to 21\% during 1985 to 2007~\cite{BP2021bp}. The share of electricity production from renewables has been continually growing since 2007. In 2020, the share of electricity production from renewables was around 29\%.}
\label{intro_renewable} % Figure 1.1
\end{figure}

Although we might not meet the complete depletion of non-renewable resources in the future 50 years, based on Hotelling’s "Economics of Exhaustible Resources", David Ricardo proposed that as the historical production stock accumulates, higher grade ores get depleted, and the producer resorts to lower grade ores, sustaining greater extraction costs~\cite{devarajan1981hotelling}. It means the extraction costs rise, and the price of the products based on ores will rise as well. 


Thus, we can assume that the price of most non-renewable resources, like oil, coal and gas, will rise since these have similar properties with ores. In fact, according to BP Statistical Review 2016~\cite{BP2016bp}, from 1987 to 2015 (from 1989 to 2015 for natural gas), the price of oil, coal and natural gas rose by approximately 36\%, 81\%, and 53\% overall (Figure~\ref{intro_fossil-fuel-price-index}). It is worth mentioning that the final price of any fuel is a complex portfolio of extraction costs, offers and demand affected by different geopolitical events.


\begin{figure}[!t]
\center
\includegraphics[scale=0.12]{img/intro_fossil-fuel-price-index.png}
\longcaption{Fossil fuel price index, 1987 to 2015.}{Prices of different fossil fuels rose from 1987 to 2015 overall~\cite{BP2016bp}.}
\label{intro_fossil-fuel-price-index} % Figure 1.2
\end{figure} 


In the past 650,000 years, there were seven-cycle glaciers to advance and retreat. However, the climate in the past 70 years has been changed differently from other periods~\cite{parmesan2003globally}. It has already had effects on the environment around us. Glaciers are shrinking, and ice is breaking up earlier on the lakes and rivers. Most climate scientists agree that it is human activities that cause global warming~\cite{epic337530}. The climate on the earth is changing throughout history.\\

As we know, atmospheric \ce{CO2} is a significant component of the atmosphere. However, atmospheric \ce{CO2} had never been above 300 parts per million until 1950 (Figure~\ref{intro_nasa_co2}). 

The atmospheric concentration of \ce{CO2} has been risen for around 36\% from 1914 to 2018 (Figure~\ref{intro_co2-concentration-long-term}).
More than a third has increased atmospheric \ce{CO2} concentration since the Industrial Revolution began~\cite{epic337530}. More importantly, atmospheric \ce{CO2} has exceeded the highest level in the past 400,000 years (Figure~\ref{intro_nasa_co2}), and it was 408.52 ppm in 2018 (Figure~\ref{intro_co2-concentration-long-term}).


\begin{figure}
\center
\includegraphics[scale=0.28]{img/intro_nasa_co2.jpeg}
\longcaption{The evidence that atmospheric \ce{CO2} has increased since the Industrial Revolution began}{\label{intro_nasa_co2} The evidence that atmospheric \ce{CO2} has increased since the Industrial Revolution began. Image courtesy: https://climate.nasa.gov/evidence} % 1950

\includegraphics[scale=0.12]{img/intro_co2-concentration-long-term.png}
\longcaption{Atmospheric \ce{CO2} concentration.}{Atmospheric \ce{CO2} concentration in 1914: 300.17ppm; Atmospheric \ce{CO2} concentration in 2018: 408.52ppm.}
\label{intro_co2-concentration-long-term}
\end{figure}




% Page 5
\newpage
In summary, the rising energy demand and lack of energy supply may cause a short-term energy crisis. The finite resources of fossil energy will drive the price of electricity and consumer goods to rise. Moreover, the burning of fossil energy will cause excessive emissions of greenhouse gases, leading to global warming. However, a door opens for cheaper but unpredictable renewable energy due to more expensive fossil fuels. According to Paris Agreements~\cite{unies2015accord}, it is vital to reduce the emissions of fossil energy on a large scale in every sector of human activities. To respond to climate change, supporting the United Nations Sustainable Development Goals and taking urgent action to address climate change and its impact, the International Maritime Organization has formulated a timetable to reduce greenhouse gas emissions from international shipping.~\cite{joung2020imo} It pointed out that between 2030 and 2050, the carbon intensity of the fleet will be reduced by at least 70\%~\cite{joung2020imo}. Before 2050, the total annual greenhouse gas emissions will be reduced by at least 50\%, which requires a reduction of approximately 85\% of carbon dioxide per ship~\cite{joung2020imo}. 

\subsection{Small-Boat Fleet and Emission Inventory}
Emission inventories for small vessels, including small fishing vessels are not developed for all years hence it is necessary to make the assumption of the emission growth for this class. In 2018, total shipping \ce{CO2} emissions increased to 1056 million tonnes compared to 962 million tonnes in 2012~\cite{IMO2021Fourth}.
In 2016, total \ce{CO2} emissions of the industrial fishing sector were 159 million tonnes, and the small-scale fishing sector emitted 48 million tonnes~\cite{GREER2019103382}. Suppose the increase rate of \ce{CO2} in 2018 was the same as the rate in 2016, and the ratio of the number of small boats and the number of boats can be approximate as the ratio of the number of small fishing boats and the number of fishing boats. Then in 2018, the total \ce{CO2} emissions for the small boat fleet can be calculated as 318.8 million tonnes.

Small vessels are classified as those smaller than 24 meters~\cite{uk2021Operational}. Knowing the shipping sector's emissions inventory can help understand what measures need to be taken to enable the industry to start the road to full decarbonization. Although it is possible to calculate large vessels from the international registry system and use the satellite data sent from the ship's transponder to account for the large vessels~\cite{IMO2021Fourth}, the small vessels depend on the national registration system, and their operation is assumed. In addition, there are many types of small vessels fleets such as machinery (e.g. fishery, people carrier, etc.), hull shape and structure, and the activities of owners and operators. The diverse small boat fleet operational profile is increasing the challenge of accurately accounting for their emission inventories.

Emissions from the global fishing industry grew by 28\% between 1990 and 2011, with a minor coinciding increase in production; however, marine fisheries are typically excluded from international assessments of \ce{CO2} or are generalized based on a limited number of case studies~\cite{parker2018fuel}. Developed economies such as the UK have a national registry~\cite{uk2021registration} that allows to have a sense of the level of small boat activity and hence infer the \ce{CO2} emissions.

However, in developing countries, it tends to be a mixed bag on the level of precision and availability. For instance, in Mexico, only fishing vessels are counted into registry~\cite{Mexico2021RegisteredVessels}. Still, it is difficult to know where they are located. Besides, the rest of the small-boat fleets are not considered. In all, Mexico does not have a regional \ce{CO2} inventory considering the small-boat fleet. Therefore, quantifying the number of small-boat fleets will allow a better precision of where the emissions are being emitted and will be the focus of understanding the emission inventory of the shipping sector.


Observing the shipping activity in the Gulf of California is essential due to its unique geographical location, conformation, and biophysical environment ~\cite{LLUCHCOTA20071, munguia2018ecological, MARINONE2012133}. In the Gulf of California, there is the largest fish producing state (Sonora) in Mexico~\cite{MELTZER2006222} and the most prominent sports fishing destination (Los Cabos, Baja) in Mexico~\cite{hernandez2012economic}. Besides, the Gulf of California has a faster shipping route to mainland Mexico than from Yucatán Peninsula.

\subsection{Bringing Convolutional Neural Networks in Satellite Images Detection}
Bringing convolutional neural networks to the field of satellite image recognition is important. First, the threshold of satellite image recognition is gradually decreasing. As the quality and quantity of global satellite images improve, obtaining the same or even better detection results with reduced parameters and complexity of convolutional neural networks is possible. Second, satellite image recognition is not new. A current PhD at the University of Texas at Austin focuses on using machine learning and remote sensing imagery to discover undersea shipwrecks ~\cite{character2021archaeologic}. However, as Figure~\ref {fig:shipwrecks} shows, discovering submarine wrecks does not require algorithms to describe the location and size of the wreck very precisely, as opposed to precisely identifying and measuring the length of a small boat. Therefore, in this thesis, accurately measuring the dimensions of small ships is definitely a challenge and a rewarding thing to do.

\begin{figure}[!t]
    \centering
    \includegraphics[scale=0.5]{img/shipwrecks.png}
    \caption{The hillshade image of the sonar or lidar output by the model.}
    \label{fig:shipwrecks}
\end{figure}


\newpage
\section{Research Questions}
This article aims to determine whether image recognition such as convolutional neural networks can be used to successfully find boats with a length of 24 meters or less on the sea surface, and classify the boats according to their attributes.

Although the object detection technology based on convolutional neural networks is now mature, it is not easy to measure accurately to the centimeter level. Should convolutional neural network or image recognition technology be an effective solution? If effective, can this method be extended to a larger sea area to detect more different types of boats? The research questions to be completed in this paper are detailed as follows:

\begin{enumerate}[(a)]

    \item What is the small-boat fleet numbers and composition around the Gulf of California?
    
    \item How accurate is the image machine-learning algorithm that recognises the small boat?
    
    \item How this algorithm can be scaled up for the rest of Mexico/west coast of the U.S.A.?
\end{enumerate}



\newpage
\section{Thesis Outline}
Following the three research questions that we just discussed, this thesis consists of four parts --- \sys{Chapter 2 Literature Review}, \sys{Chapter 3 Methodology}, \sys{Chapter 4 Results}, and \sys{Chapter 5 Conclusions and Future Work}.

\begin{description}
    \item In Chapter~\ref{chap:2}, I will begin with an overview of recent developments in identifying small boats and shipping decarbonisation. Next, I will briefly discuss developing convolutional neural networks and algorithms for cloud removal for satellite images.

    \item In Chapter~\ref{chap:3}, I will talk about the algorithms I used in this project and their mathematical foundations, such as convolutional neural networks. Then, I will discuss the data sources. For example, how to improve the original data specification to fit an existing deep learning framework. Likewise, I will briefly introduce the advantages of GPUs. This will enable the reader to understand the mathematics behind this decision to use GPUs. Then, I will talk about everything about detection and classification. First, I will start by defining the scope of recognition. I will explain why I selected only some of the cities in the Gulf of California. Then, I will discuss the quality of the data provided by Google Earth Pro and the principles and effects of sharpening images using image kernels. Finally, I will explain the principle of detecting the length of small boats and the thinking behind classifying small boats. At the end of this chapter, I will include a workflow on the algorithm.
    
    \item In Chapter~\ref{chap:4}, I will discuss the logs and the length of the detected boats when training the convolutional neural network. Importantly, I will also analyze the composition of the small boats in the three port cities and the potential reasons for this.
    
    \item In Chapter~\ref{chap:5} and Chapter~\ref{chap:6}, I will discuss a few pressing issues to be addressed in the future and summarize the conclusions reached and the progress made.
\end{description}


\newpage
\section{Contributions}
The contributions of this thesis are summarized as follows:
\begin{itemize}
    \item The accuracy of my trained object detection algorithm can reach 96\% to 98\%.

    \item I proposed and implemented an algorithm for measuring the length of an object (small boat) based on object detection, which showed excellent accuracy in tests.

    \item I proposed and implemented the algorithm to determine whether the detected small boat is a domestic recreational small boat, dividing small boats into domestic recreational small boats and fishing boats. Although the algorithm has flaws, this is an attempt to use a pure computer vision algorithm rather than a neural network in identifying objects. Nevertheless, I believe this is the first but important step in constructing a way to identify the types of small boats. Based on this, I answered questions about the composition of small boats in the Gulf of California.

    \item I developed Python scripts that can count the number of small boats, the number of large boats, and the number of small boats for home recreation. This automated the statistics and saved a great deal of time in counting.

\end{itemize}


% Literature Review
\chapter{Literature Review}
\label{chapter:Review}
%!TEX root = thesis.tex
\label{chap:2}
\section{Small Boat Fleet and Shipping Decarbonisation}
Current literature related to estimating small-scale vessels without machine learning methods includes using statistical factors or measures.~\newcite{johnson2017spatial} used Kernel Density Estimation (KDE) to distribute data on the population, the number of ships, and the average annual total catch for the entire population, and finally showed that their forecasts could accurately predict the landing of fisheries in the bay. In their paper, they explained the source of their data~\cite{Lopez-Sagastegui2017Comparing}. Lopez-Sagastegui et al. worked with fishers to generate and record information about fish catches, fishing efforts, profits, species breeding seasons, and spatial patterns of fishing activities. However, one of the points against this work is that it only focuses on fishing vessels. While these are the majority, it still misses the other ship types. Notably, the authors provided information (Figure~\ref{review_density_of_gulf}) of the density of human population and boats in the Gulf of California that can save much time in creating data sets of the Gulf of California for training machines learning models.\\

\begin{figure}[t]
\center
\includegraphics[scale=0.83]{img/review_density_of_gulf.png}
\longcaption{Map of the density of human population, boats, and fishing offices in the Gulf of California.}{\label{review_density_of_gulf} Map of the density of human population, boats, and fishing offices in the Gulf of California. Image courtesy: https://doi.org/10.1371/journal.pone.0174064.g002}
\end{figure}

~\newcite{Johansson2018ModelingOL} proposed a new model (FMI-BEAM) to describe the emissions of the leisure boat fleet in the Baltic Sea region with over 3000 dock locations, national small boat registry, AIS data and vessel survey results. However, the method cannot cover countries with no national registry for small boats. Besides, small boats are not just leisure boats.~\newcite{Ug2020EstimationOW} estimated global ship emissions with the help of data from the Automatic Identification System (AIS). They set up movement information relating to ship size and speed and meteorological and marine environmental conditions. Over 3,000,000,000 daily AIS data records from hundreds of owners and thousands of partner AIS base stations and detailed ship data. However, this method is highly dependent on AIS data which is impossible for unregistered small boats.~\newcite{Traut2013MonitoringSE, Johansson2016ACM, Mabunda2014EstimatingCD, Hensel2020GreenSU, Han2016RealtimeIA} have proposed the use of AIS to monitor the carbon emissions of the fleet as well.\\

~\newcite{Zhang2019TheSO} included unidentified vessels in the AIS-based vessel emission inventory. They developed an AIS-instrumented emissions inventory, including both identified and unidentified vessels. In particular, missing vessel parameters for unidentified vessels were estimated from a classification regression of vessels with similar vessel types and sizes in the AIS database. However, the authors do not discuss whether the regression model applies to vessels in most coastal areas of the planet. In addition, the authors do not discuss whether the vessel data in the AIS database is regionally diverse. Finally, if there is a diversity of vessels in the AIS database, the authors did not discuss whether this diversity would produce more significant errors in the predictions for small vessels in a single region (e.g. the Gulf of California, Mexico).


\section{Convolutional Neural Networks in Image Recognition}
\label{sec2.2}
Literature tends to be inaccurate for emission inventories for the small boat fleet. The above literature review has demonstrated that there is still a lot of work to understand how the small boat fleet is being operated, what fuels they are using, and the level of activity for this shipping sector. This master thesis project intends to use image recognition algorithms to identify small boats in any sea area, significantly reducing the time taken to calculate small vessel emission inventories. Besides, it will be in the national interest for the small fleet to account for and control these emissions within the powers of the state, incentivising the energy-efficient technologies and fuel change. Further, if countries are to meet their ambitious net-zero carbon emissions targets, they cannot afford to ignore the small boat fleet emission inventories that can help governments account for carbon emissions from small boats more quickly.\\

Object detection is an active topic in image recognition and computer vision. In the past few years, this topic has made significant progress. With the rise of self-driving cars and face detection, there is an increasing demand for fast and accurate objects detection. In 2012, AlexNet won the ImageNet Large-Scale Visual Recognition Challenge (ILSVRC), making convolutional neural networks the dominant mode of image recognition~\cite{krizhevsky2012imagenet}. Then,~\newcite{girshick2014rich} introduced R-CNN, which is the first CNN-based object detection method, but with higher performance.\\

Unlike other deep learning problems, the unknown number, size and categories of instances in each image in the object detection problem lead to unexplored dimensions of the model output. Therefore, adapting the classic deep learning classification model that expects a fixed output size to the object detection problem is difficult. Region-Based Convolutional Neural Networks (R-CNN) may be the way to meet this challenge.\\

Since its inception, Region-Based Convolutional Neural Networks (R-CNN) has gone through many iterations: R-CNN, Fast R-CNN, Faster R-CNN and Mask R-CNN~\cite{girshick2014rich, girshick2015fast, ren2015faster, he2017mask}. To avoid computationally expensive, pixel-level classification and object search, the original R-CNN method applied a non-deep learning algorithm called Selective Search to obtain approximately 2000 region suggestions~\cite{uijlings2013selective, girshick2014rich}. Then, with the help of a modified version of the AlexNet model, features are extracted from the proposed region. The objects in the image have different scales and sizes, so their corresponding image crops need to be warped to fit the following CNN input size. Then, the support vector machine (SVM) classifier uses the extracted features for classification, and the linear regression predicts the offset of the bounding box, thereby tightening the bounding box around the object~\cite{girshick2014rich}.\\

R-CNN is an intuitive architecture that has achieved high accuracy in object detection. However, since the reasoning time for each image is about 30 seconds, this model is very inefficient for applications that require real-time prediction~\cite{huang2017speed}. Fast R-CNN performs feature extraction on the image before generating the region suggestion. The region suggestion is generated based on the final feature map instead of the original image itself, significantly improving the training and inference speed. Therefore, only one CNN forward channel is calculated on the entire image, instead of about 2000 channels as it is now. Although Fast R-CNN does improve efficiency, selective search to generate region recommendations is the most computationally intensive part of the architecture, which forms a bottleneck in network inference and training time. Faster R-CNN alleviates this problem by replacing the selective search algorithm with a neural network called a region proposal network (RPN). The network simultaneously predicts the boundaries of objects and the classification between the two types of objects at each location and the background~\cite{ren2015faster}. The rest of the architecture is similar to that proposed by Fast R-CNN.
A recently proposed architecture, Mask R-CNN~\cite{he2017mask}, can perform object detection in addition to instance segmentation by adding a fully convolutional network (FCN) branch to the Faster R-CNN architecture. Instance segmentation can be defined as demarcating and classifying object instances belonging to different categories in the image.\\

However, even traditional CNNs can be very useful for large-scale image recognition.~\newcite{Simonyan2015VeryDC} in the University of Oxford, and Google DeepMind researched the effect of convolutional network depth on its accuracy in the large-scale image recognition setting. Their research found out that even they used very small (3x3) convolution filters, a significant improvement can be achieved by pushing the depth to 16 to 19 weight layers.\\


\section{Noise Removal for Image in the Shipping Sectors or Similar Applications}
Satellite images often have targets that should not be there, such as shadows cast by water on the sea surface due to sunlight or clouds in the atmosphere. These noises can make the training data inaccurate and often cause problems for the correctness of the model.~\newcite{He2009SingleIH} proposed a simple but effective image prior-dark channel before removing haze from a single input image. The dark channel prior is a kind of statistics of outdoor haze-free images. It is based on critical observation-most local patches in outdoor haze-free images contain some pixels whose intensity is very low in at least one colour channel. Using this prior to the haze imaging model, the thickness of the haze can be estimated, and a high-quality haze-free image can be recovered. Moreover, a high-quality depth map can also be obtained as a byproduct of haze removal.\\

% Methodology
\chapter{Methodology}
\label{chapter:Methodology}
\section{Target Areas in the Gulf of California: Santa Rosalia, Loreto, Guaymas}


\section{Train Custom Data}
\subsection{Data Source, Labelling, Auto-orient and Resize}
\subsection{Deep Learning Architecture: CNN/ConvNet}
\subsection{Training Models with YOLOv5 (Small, Medium, Large, XLarge)}


\section{Detection and Classification}
\subsection{Image Kernel and Detecting Small Boats}
\subsection{Detecting the Length of Small boats}
\subsection{An Classification of Small Boats and Big Boats}
\subsection{An Classification of Entertainment Boats and Fishing Boats}

\section{Statistical Algorithm Design}

\section{Workflow}

% Results
\chapter{Results}
\label{chapter:Results}

\section{Train Custom Data}
\subsection{Weights and Biases Logging}
\subsection{Local Logging}
\subsection{The Best Weight and Testing Results}

\section{Detection and Classification}

\section{Differences in Vessel Composition in Santa Rosalia, Loreto, Guaymas}

% Conclusion
\chapter{Conclusions and Future Work}
\label{chapter:Conclusions_and_Future_Work}
 % 1. Draw together all the work and results in the above report and to present a succinct summary.
 \section{Draw together all the work and results}
 
 % 2.Future Work: 本来应该分类成「更加精确的分类」Leisure boat / Shipping boat / family boat... 需要更多清晰 (分辨率 + 颜色) 的测试数据集... 在训练数据的时候就很清晰要分类成什么.....) + 扩展到更多的地区
 \section{Future Work}
 
 % 3. 画大饼 propose
\section{Pathway + Propose}
 
 % 4. Include a brief reflection on what I have learnt from undertaking my project as far as project management is concerned.
 
\section{Brief Reflection}

% Bibliography
\bibliographystyle{bartlett_harvard_ref}
\bibliography{ref}

\end{document}
