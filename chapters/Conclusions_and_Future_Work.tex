\section{Conclusions}
 In this thesis, I introduce the reader to satellite image recognition and how I can make a modest contribution to the development of the field.\\

In Chapter~\ref{chap:2}, I review the traditional research literature on estimating the number of small boats. Researchers use kernel density estimation (KDE) to distribute population and vessel numbers and ultimately show that their predictions can accurately predict the number of vessels in the Gulf fisheries. Some scholars then focused on describing the combination of national small vessel registries and surveyed vessel discharge results. However, the methodology does not cover states without a national small boat registry, and small boats are not just recreational vessels.\\

Obviously, these traditional approaches do not extract information well from a large amount of data. Although as early as the 1990s, researchers such as Yann LeCun realized that convolutional neural networks could recognize images. However, with the exponential growth in computing power in the last five years, scientists and engineers have only been able to apply convolutional neural networks to topics such as large-scale image recognition and target detection. Since 2012, the field has been reinvented by creating large-scale supervised datasets and developing neural target detection network models. Although it has only been nine years since then, the field has evolved at an amazing rate. Innovations in building better datasets and more effective models have alternated, and both have contributed to the field's growth.\\

In Chapter~\ref{chap:3}, I first explain how convolutional neural networks can detect a letter `X' and describe the mathematical principles involved. Next, I introduce the two most commonly used neural network frameworks today: Faster RCNN and YOLO and cite Dwivedi's work to illustrate to the reader the advantages of the YOLO model in recognizing small targets advantage of higher accuracy in recognizing targets in videos. Next, I show the data needed in training the model and the preparation needed to train the model faster. Finally, to speed up the model's training, I use a GPU to train the model and use Google Colab, Google Drive and GitHub as my testing and development tools.\\


In the subsequent section, I highlight the inequality in the amount of data available on satellite images for the Gulf of California. Thus, I took the approach of selecting one of the cities with a large amount of satellite data every year and then analyzing it. However, even this does not avoid the fact that the quality of the satellite images in 2019 is worse compared to the following two years. In response, I took the approach of sharpening the images to bring out the details of the images. The model is thus able to identify the target better.\\

Then, I designed algorithms to detect the length of small boats. Since I did not have access to the zoom scale of Google Earth Pro, I had to define a zoom scale myself to find the relationship between the length of the real boat and the boat in the image. Finally, by using the same eye altitude and resolution for all the photos in the test dataset, the algorithm can automatically detect the boat's length. In the test, the boat's length detected by the algorithm is similar to the length measured from Google Earth Pro.\\

Finally, I show the detection results that can combine open high-resolution satellite data and convolutional neural networks. The results are not as high as the ideal recognition rate, but such detection results are still acceptable due to the inferior quality of the monitoring data. The results show a divergence. On the one hand, there is a large port like Guaymas with many large and small vessels. On the other hand, small ports like Loreto and Santa Rosalia have almost no big ships and fewer small ships. But it is interesting to note that if there are small boats in Guaymas, Loreto and Santa Rosalia, they are almost certainly small boats for family use for recreation. The analysis of satellite images shows that most small boats are docked in the harbour, which means it is rare to see a large number of small boats floating on the sea. Another interesting point is that the number of small boats identified increases as the year increases. Considering that most of the boats are in the harbour and that the models used for the tests are identical, the only difference is that the detail of the satellite images is better realizable each year, i.e., each year we have an image with higher quality. Then, it is reasonable to believe that the increase in the quality of satellite photos can improve the quality of target detection.



 
 % 2.Future Work: 本来应该分类成「更加精确的分类」Leisure boat / Shipping boat / family boat... 需要更多清晰 (分辨率 + 颜色) 的测试数据集... 在训练数据的时候就很清晰要分类成什么.....) + 扩展到更多的地区
 \section{Future Work}
Although the location and size of small boats have been successfully identified, and the boats have been successfully but roughly classified into domestic recreational and fishing categories, more work is still needed to refine and improve them. In the future,

\begin{enumerate}
    \item Can data providers provide scholars with clear and close satellite imagery for free? Data is one of the triads of artificial intelligence and also the most overlooked factor. In fact, AI algorithms are highly dependent on data. It is as if the algorithms are always hungry, and all need a constant stream of data to feed their hunger. For target detection algorithms, poor image detail representation means less high-quality data. For the algorithm, fewer data often brings poor output results. Therefore, both the data used in building the model and the data used for detection should be kept at the highest level. Otherwise, training AI models loses its relative meaning.
    
    \item Will Google Earth Pro provide a deep integration tool about the zoom scale in the future? Google Earth Pro provides very many images with details. However, Google Earth Pro does not offer similar integrated tools for zooming rulers as Google Maps does. If such a scaling environment were available, it would mean that every photo taken on Google Earth Pro would contain its scaling ratio. This means that we do not need to define a fixed zoom scale. It would be easy to know the object's length on each screenshot, regardless of whether the eye altitude is 200 meters or not. This leads to the fact that when some giant cargo ships or cruise ships need to be detected, we can reduce the scale to get an overall picture of the giant ship. When some tiny ships (e.g., a small ship under 5 meters in length) need to be inspected, we can zoom in to get the overall shape of the small ship.
    
    \item Will small boats have a more precise classification in the future? Using colour to distinguish whether a boat is a recreational boat or a fishing boat may not be a solution that is acceptable to everyone. However, to solve this problem with a purely deep learning approach, one needs to train the data with all the types of boats that one wants to detect. However, doing the data classification and training in a realistic environment, i.e. Google Earth Pro, would be labour-intensive and costly. Furthermore, when looking for the classification category and the data under this category, is it possible to guarantee that the object‘s environment is the same as the object's environment under other categories due to the AI fairness principle? For example, is it possible to guarantee that the proportion of large cruise ships appearing on the beachside is the same as that of small recreational boats appearing on the beachside? The reason for this is that we don't want all large cruise ships to be offshore and all small recreational boats to be on the beach. If this is the case, the deep neural network does not need to know the contours or characteristics of the different boats. It just needs to analyze the background of the boats (i.e., the colour of the sea level, the colour of the beach, etc.) to predict whether a large cruise ship is likely to be a large cruise ship in the first place. One of the potential solutions to this problem is using computer simulation techniques to create CAD models of various ships. These CAD ships are then placed evenly into a variety of different background environments. Finally, these data are trained with deep neural networks, and then the trained models are put into real-life situations to detect some real ships.
    
    \item Will it be possible in the future to quickly scale up the algorithm to other regions? As well, will it be possible to analyze real-time satellite video in the future? Again, the key to expanding the algorithm to other regions and analyzing real-time satellite video is to have fast and timely access to the data in the region. Frankly, it is whether the data providers can provide the data in the region. With real-time satellite video, it will be possible to get a better picture of maritime traffic in any area to monitor and control the carbon emissions of shipping in that area.
    
    \item Is it possible to add more recognition objects, such as harbour and sea ripples, to prevent the recognition of harbour or sea ripples as small boats? The purpose of this is that once the algorithm can successfully identify the harbour or sea ripples, they will not be identified as small boats.
    
    \item Can the timeframes be better? It is a question of optimization, a trade-off question on cloud storage, a question of the algorithm's speed, and a question of the accuracy of the result. Besides, can the algorithm distinguish the same boat?
\end{enumerate}
 
 
 
 % 4. Include a brief reflection on what I have learnt from undertaking my project as far as project management is concerned.
 
\section{Brief Reflection}